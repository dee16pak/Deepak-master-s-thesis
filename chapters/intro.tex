\chapter{Introduction}
\label{ch:int}

The term smart contracts, in an informal definition, can be interpreted as legal piece of agreement between two parties without the involvement of third party inter mediators. Smart contracts represent a next step in the progression of blockchains from a financial transaction protocol to an all-purpose utility. They are pieces of software, not contracts in the legal sense, that extend blockchains utility from simply keeping a record of financial transaction entries to automatically implementing terms of multiparty agreements. Smart contracts are executed by a computer network that uses consensus protocols to agree upon the sequence of actions resulting from the contract’s code. The result is a method by which parties can agree upon terms and trust that they will be executed automatically, with reduced risk of error or manipulation.\cite{deloitte} \\
Smart Contracts were first introduced by an American Scientist and Cryptographer, Nick Szabo, in the early 1990s\cite{smartcontract}. However, smart contracts gained popularity with the introduction of Ethereum, which uses the Solidity language to program the contracts. Bitcoin also supports smart contracts, but one must know opcode programming to use it. This makes smart contracts usage in Bitcoin very limited\cite{forbes}. But unlike Bitcoin, Ethereum can do much more. Developers can use Ethereum to build new kind of applications(dApps). The Ethereum community is the largest and most active blockchain community in the world. Thousands of developers all over the world are developing applications, and inventing new kinds of applications, many of which we can used today. Hence smart contracts on ethereum are increasing with a faster pace. Infact, the number of smart contracts deployed on the Ethereum network reached near 2M in March 2020\cite{cointelegrah}. The total supply of Ether has crossed 100M mark in April 2020\cite{ether}.\\
Smart contracts are immutable as long as blockchain integrity is not compromised. Since we cannot change smart contracts once deployed, any security vulnerabilities due to coding logic in the smart contracts will be visible to the public and cannot be patched. Hackers/attackers have exploited many such vulnerabilities in the past. Due to these exploitation's, damage of some million dollars worth ether happened already. The bigger issue is the trust of public. Suppose a company/developer deployed a faulty contract on ethereum and it is attacked, then the people no longer trust that company/developer. Hence during coding the developer have to be very careful and have to save his/her smart contract from exploitation at-least because of the known vulnerabilities. To facilitate the developers, a good number of tools using various approaches, eg-static analysis, symbolic analysis etc are already built to check for various vulnerabilities. Every tools looks for specific set of vulnerabilities in the smart contracts. This thesis focused on the DoS vulnerabilities of the smart contract in our tool. The main goal of this thesis is to build a symbolic analysis tool which will check for the Denial of Service vulnerabilities in the smart contracts.\\
The thesis is divided as follows: Chapter \href{ch:bcg}{2} provides background for various concepts necessary to understand the thesis work. Chapter \href{ch:DosVul}{3} introduces various DoS vulnerabilities found in the past and existing related work related to denial of service. Chapter \href{ch:symA}{4} introduces the symbolic execution approach for the Dos vulnerabilities and various DoS vulnerabilities patterns which are summarised from the DoS vulnerabilities explained in Chapter \href{ch:DosVul}{3}. Chapter \href{ch:work}{5} explains the working of the tool and explains all components of the tool in detail. Chapter \href{ch:result}{6} shows the result of the smart contract and analysis on false positives of the tool. Chapter \href{ch:conclusion}{7} summarises the thesis and the future scope of the thesis.